% Created 2017-11-05 Sun 20:26
\documentclass[11pt]{article}
\usepackage[utf8]{inputenc}
\usepackage[T1]{fontenc}
\usepackage{fixltx2e}
\usepackage{graphicx}
\usepackage{longtable}
\usepackage{float}
\usepackage{wrapfig}
\usepackage{rotating}
\usepackage[normalem]{ulem}
\usepackage{amsmath}
\usepackage{textcomp}
\usepackage{marvosym}
\usepackage{wasysym}
\usepackage{amssymb}
\usepackage{hyperref}
\tolerance=1000
\author{Nitin John Raj}
\date{\today}
\title{Implementation of Simply Typed Lambda Calculus}
\hypersetup{
  pdfkeywords={},
  pdfsubject={},
  pdfcreator={Emacs 24.5.1 (Org mode 8.2.10)}}
\begin{document}

\maketitle
\tableofcontents

In this chapter, we will attempt to build a system for simply typed lambda calculus as defined in \href{../typed-lambda-calculus.org}{Simply Typed Lambda Calculus}. We will be using the de Brujin method of evaluation. We shall only be dealing with abstract syntax here - the parser from concrete to abstract syntax is beyond the scope of this chapter, and teaches no necessary concepts with respect to simply typed lambda calculus.

\section{Abstract Syntax}
\label{sec-1}
\subsection{Types}
\label{sec-1-1}
\subsubsection{Documentation}
\label{sec-1-1-1}
Our model of lambda calculus allows us to use types. We explicitly annotate our variables with these types.

The types we shall support in this model are booleans and integers, and functions.

\subsubsection{Code}
\label{sec-1-1-2}
\begin{verbatim}
data Type = Boolean
	  | Any
	  | Function Type Type
	  deriving (Eq, Show)
\end{verbatim}

\subsection{Standard Lambda Calculus}
\label{sec-1-2}
\subsubsection{Documentation}
\label{sec-1-2-1}
The terms of standard lambda calculus are:
\begin{itemize}
\item variables
\item abstractions
\item applications
\end{itemize}

Furthermore, we also enrich our model with primitive booleans and expressions:
\begin{itemize}
\item true
\item false
\item if \_ then \_ else \_
\end{itemize}

Valid variable identifiers can be represented by strings. 

\subsubsection{Code}
\label{sec-1-2-2}
\begin{verbatim}
type Identifier = String

data Term = Var Identifier
	  | Abs Identifier Type Term
	  | App Term Term
	  | PrimTrue
	  | PrimFalse
	  | IfThenElse Term Term Term
	  deriving (Eq, Show)
\end{verbatim}

\subsection{De Brujin Representation}
\label{sec-1-3}
\subsubsection{Documentation}
\label{sec-1-3-1}
De Brujin terms are an alternate representation of lambda calculus. Here, the variable identifiers are replaced with the \emph{static distance} from their lambda binding. However, we shall preserve the identifier bound to each abstraction in this AST, as it makes evaluation easier.

\subsubsection{Code}
\label{sec-1-3-2}
\begin{verbatim}
data DeBrujinTerm = Index Int 
		  | Lambda Identifier Type DeBrujinTerm
		  | Application DeBrujinTerm DeBrujinTerm
		  | True'
		  | False'
		  | If DeBrujinTerm DeBrujinTerm DeBrujinTerm
		  deriving (Eq, Show)
\end{verbatim}


\section{Free Variables}
\label{sec-2}
\subsection{Documentation}
\label{sec-2-1}
A free variable is one that is not bound to an abstraction. Here, we define a function to find free variables in our standard AST.

\subsection{Code}
\label{sec-2-2}
\begin{verbatim}
fv :: Term -> [Identifier]
fv (Var x) = [x]
fv (Abs x q t) = (fv t) \\ [x]
fv (App s t) = (fv s) `union` (fv t)
fv (IfThenElse r s t) = (fv r) `union` (fv s) `union` (fv t)
fv _ = []
\end{verbatim}


\section{Context}
\label{sec-3}
\subsection{Documentation}
\label{sec-3-1}
The context helps map variables to data about the variables. We will be using the context to map identifiers to their corresponding de Brujin index, and to the type they're annotated with.

On the context, we define certain functions:

\begin{itemize}
\item \[typeOf: Int \times Context \to Type\]
\item \[indexOf: String \times Context \to Int\]
\end{itemize}

\subsection{Code}
\label{sec-3-2}
\begin{verbatim}
type Context = [(Identifier, Type)]

indexOf :: Identifier -> Context -> Int
indexOf x context = case (findIndex (\y -> fst y == x) context) of
		      Just n -> n
		      Nothing -> error "Identifier does not exist in context!"

getType :: Int -> Context -> Type
getType i context = snd (context !! i)

getIdentifier :: Int -> Context -> Identifier
getIdentifier i context = fst (context !! i)
\end{verbatim}


\section{Parser}
\label{sec-4}
\subsection{Documentation}
\label{sec-4-1}
For our implementation, we first need a parser from our standard AST model to our de Brujin terms. 

\subsection{Code}
\label{sec-4-2}
\begin{verbatim}
convert :: Term -> (DeBrujinTerm, Context)
convert x =
  let
    aux :: Context -> Term -> DeBrujinTerm
    aux context (Var x) = Index (indexOf x context)
    aux context (Abs x q t) = Lambda x q (aux ((x, q) : context) t)
    aux context (App s t) = let convert' = aux context
			    in Application (convert' s) (convert' t)
    aux _ PrimTrue = True'
    aux _ PrimFalse = False'
    aux context (IfThenElse r s t) = let convert' = aux context
				     in If (convert' r) (convert' s) (convert' t)

    buildContext :: Term -> Context
    buildContext x = map (\x -> (x, Any)) (fv x)

    context :: Context
    context = buildContext x
  in
    (aux context x, context)
\end{verbatim}


\section{Deparsing}
\label{sec-5}
\subsection{Documentation}
\label{sec-5-1}
We require a parser to convert our standard AST to our de Brujin AST for evaluation. Once evaluation is done, we would like to deparse the de Brujin AST back to the standard one.

\subsection{Code}
\label{sec-5-2}
\begin{verbatim}
convertBack :: Context -> DeBrujinTerm -> Term
convertBack context (Index k) = Var (getIdentifier k context)
convertBack context (Lambda x q t) = Abs x q (convertBack ((x, q) : context) t)
convertBack context (Application s t) = App (convertBack context s) (convertBack context t)
convertBack _ True' = PrimTrue
convertBack _ False' = PrimFalse
convertBack context (If t1 t2 t3) = let convert' = convertBack context
				    in IfThenElse (convert' t1) (convert' t2) (convert' t3)
\end{verbatim}


\section{Shifting}
\label{sec-6}
\subsection{Documentation}
\label{sec-6-1}
Shifting is an operation on de Brujin terms that is used in beta reduction. The d-place shift at a cutoff of c is defined as follows:

\[\uparrow^d_c(k) = k\], \[k < c\]
\[\uparrow^d_c(k) = k + d\], \[k \ge c\]
\[\uparrow^d_c(\lambda.\ t_1) = \lambda.\ \uparrow^d_{c + 1}(t_1)\]
\[\uparrow^d_c(t_1\ t_2) = \uparrow^d_c(t_1)\ \uparrow^d_c(t_2)\]

\subsection{Code}
\label{sec-6-2}
\begin{verbatim}
(-^-) :: Int -> Int -> (DeBrujinTerm -> DeBrujinTerm)
(d -^- c) (Index k)
  | k < c = Index k
  | otherwise = Index (k + d)
(d -^- c) (Lambda x q t) = Lambda x q ((d -^- (c + 1)) t)
(d -^- c) (Application s t) = let shift = d -^- c
			      in Application (shift s) (shift t)
(_ -^- _) literal = literal
\end{verbatim}


\section{Substitution}
\label{sec-7}
\subsection{Documentation}
\label{sec-7-1}
Substitution is defined as follows:

\[[i \mapsto s]i = s\]
\[[i \mapsto s]k = k\]
\[[i \mapsto s](\lambda.\ t_1) = \lambda.\ [i + 1 \mapsto s]\uparrow^1(t_1)\]
\[[i \mapsto s](t_1\ t_2) = ([i \mapsto s]t_1\ [i \mapsto s]t_2)\]

\subsection{Code}
\label{sec-7-2}
\begin{verbatim}
(-:>) :: Int -> DeBrujinTerm -> (DeBrujinTerm -> DeBrujinTerm)
(i -:> s) (Index k)
  | i == k = s
  | otherwise = Index k
(i -:> s) (Lambda x q t) = Lambda x q (((i + 1) -:> s) ((1-^-1) t))
(i -:> s) (Application r t) = Application ((i -:> s) r) ((i -:> s) t)
(_ -:> _) literal = literal
\end{verbatim}


\section{Values}
\label{sec-8}
\subsection{Documentation}
\label{sec-8-1}
We need to define values in our interpreter. Values include:
\begin{itemize}
\item abstractions
\item variables
\item literals
\item applications where all terms are values and where the first term is not an abstraction
\end{itemize}

\subsection{Code}
\label{sec-8-2}
\begin{verbatim}
isValue :: DeBrujinTerm -> Bool
isValue (Application (Lambda _ _ _) _) = False
isValue (Application x y) = (isValue x) && (isValue y)
isValue (If _ _ _) = False
isValue _ = True
\end{verbatim}


\section{Typechecking}
\label{sec-9}
\subsection{Documentation}
\label{sec-9-1}
Typechecking is done using a direct application of the inversion theorem.

\begin{itemize}
\item \[\Gamma \vdash x: T \implies x:T \in \Gamma\]

\item \[\Gamma \vdash (\lambda x:T_1.\ t): R \implies \exists T_2(R = T_1 \to T_2 \iff \Gamma, x:T_1 \vdash t:T_2)\]

\item \[\Gamma \vdash (t_1\ t_2): R \implies \exists T_1((\Gamma \vdash t_1: T_1 \to R) \wedge (\Gamma \vdash t_2: T_2))\]

\item \[\Gamma \vdash true: R \implies R = Bool\]

\item \[\Gamma \vdash false: R \implies R = Bool\]

\item \[\Gamma \vdash (if\ t_1\ then\ t_2\ else\ t_3): R \implies \Gamma \vdash t_1: Bool, \Gamma \vdash t_2: R, \Gamma \vdash t_3: R\]
\end{itemize}

\subsection{Code}
\label{sec-9-2}
\begin{verbatim}
typeOf :: Term -> Maybe Type
typeOf term =
  let
    aux :: DeBrujinTerm -> Context -> Maybe Type
    aux (Index k) context = Just (getType k context)
    aux (Lambda x q t) context =
      let
	context' = (x, q) : context
	q' = aux t context'
      in
	case q' of
	  Nothing -> Nothing
	  Just r -> Function q r
    aux (Application t1 t2) context =
      let
	q1' = aux t1 context
	q2' = aux t2 context
      in
	case (q1', q2') of
	  (Nothing, _) -> Nothing
	  (_, Nothing) -> Nothing
	  (Just (Function q11 q12), Just q2) -> if (q11 == q2)
						 then Just q12
						 else Nothing
	  (_, _) -> Nothing
    aux True' _ = Just Boolean
    aux False' _ = Just Boolean
    aux (If t1 t2 t3) context=
      case (aux t1 context) of
	Just q1 -> if (q1 == Boolean)
		   then case (aux t2 context, aux t3 context) of
			  (Just q2, Just q3) -> if (q2 == q3)
						then Just q2
						else Nothing
			  (_, _) -> Nothing
		   else Nothing
	Nothing -> Nothing

    (t, context) = convert term
  in
    aux t context
\end{verbatim}


\section{Beta Reduction}
\label{sec-10}
\subsection{Documentation}
\label{sec-10-1}
Beta reduction by call-by-value strategy follows the following rules:

E-APP$_{\text{1}}$: \[\frac{t_1 \to v_1}{t_1\ t_2 \to v_1\ t_2}\]

E-APP$_{\text{2}}$: \[\frac{t_2 \to v_2}{v_1\ t_2 \to v_1\ v_2}\]

E-APP$_{\text{ABS}}$: \[(\lambda.\ t_{1,2})\ v_2 = \uparrow^{-1}([0 \mapsto \uparrow^1(v_2)]t_{1, 2})\]

\subsection{Code}
\label{sec-10-2}
\begin{verbatim}
evaluate :: Term -> (Term, Type)
evaluate term =
  let
    aux :: DeBrujinTerm -> DeBrujinTerm
    aux (Application s t)
      | not (isValue s) = Application (aux s) t
      | not (isValue t) = Application s (aux t)
      | otherwise = case s of
		      Lambda x q e -> ((-1)-^-0) ((0 -:> ((1-^-0) t)) e)
		      _ -> Application s t
    aux (If r s t)
      | predicate == True' = aux s
      | predicate == False' = aux s
      | otherwise = error "Non-boolean in test position!"
      where predicate = aux r
    aux x = x

    (t, context) = convert term
    q' = typeOf term
  in
    case q' of
      Just q -> (q, aux t)
      Nothing -> error "Not a well typed term!"
\end{verbatim}


\section{Putting it Together}
\label{sec-11}
\begin{verbatim}
import Data.List
\end{verbatim}

\begin{verbatim}
import Data.List

data Type = Boolean
	  | Any
	  | Function Type Type
	  deriving (Eq, Show)

type Identifier = String

data Term = Var Identifier
	  | Abs Identifier Type Term
	  | App Term Term
	  | PrimTrue
	  | PrimFalse
	  | IfThenElse Term Term Term
	  deriving (Eq, Show)

data DeBrujinTerm = Index Int 
		  | Lambda Identifier Type DeBrujinTerm
		  | Application DeBrujinTerm DeBrujinTerm
		  | True'
		  | False'
		  | If DeBrujinTerm DeBrujinTerm DeBrujinTerm
		  deriving (Eq, Show)

fv :: Term -> [Identifier]
fv (Var x) = [x]
fv (Abs x q t) = (fv t) \\ [x]
fv (App s t) = (fv s) `union` (fv t)
fv (IfThenElse r s t) = (fv r) `union` (fv s) `union` (fv t)
fv _ = []

type Context = [(Identifier, Type)]

indexOf :: Identifier -> Context -> Int
indexOf x context = case (findIndex (\y -> fst y == x) context) of
		      Just n -> n
		      Nothing -> error "Identifier does not exist in context!"

getType :: Int -> Context -> Type
getType i context = snd (context !! i)

getIdentifier :: Int -> Context -> Identifier
getIdentifier i context = fst (context !! i)

convert :: Term -> (DeBrujinTerm, Context)
convert x =
  let
    aux :: Context -> Term -> DeBrujinTerm
    aux context (Var x) = Index (indexOf x context)
    aux context (Abs x q t) = Lambda x q (aux ((x, q) : context) t)
    aux context (App s t) = let convert' = aux context
			    in Application (convert' s) (convert' t)
    aux _ PrimTrue = True'
    aux _ PrimFalse = False'
    aux context (IfThenElse r s t) = let convert' = aux context
				     in If (convert' r) (convert' s) (convert' t)

    buildContext :: Term -> Context
    buildContext x = map (\x -> (x, Any)) (fv x)

    context :: Context
    context = buildContext x
  in
    (aux context x, context)

convertBack :: Context -> DeBrujinTerm -> Term
convertBack context (Index k) = Var (getIdentifier k context)
convertBack context (Lambda x q t) = Abs x q (convertBack ((x, q) : context) t)
convertBack context (Application s t) = App (convertBack context s) (convertBack context t)
convertBack _ True' = PrimTrue
convertBack _ False' = PrimFalse
convertBack context (If t1 t2 t3) = let convert' = convertBack context
				    in IfThenElse (convert' t1) (convert' t2) (convert' t3)

(-^-) :: Int -> Int -> (DeBrujinTerm -> DeBrujinTerm)
(d -^- c) (Index k)
  | k < c = Index k
  | otherwise = Index (k + d)
(d -^- c) (Lambda x q t) = Lambda x q ((d -^- (c + 1)) t)
(d -^- c) (Application s t) = let shift = d -^- c
			      in Application (shift s) (shift t)
(_ -^- _) literal = literal

(-:>) :: Int -> DeBrujinTerm -> (DeBrujinTerm -> DeBrujinTerm)
(i -:> s) (Index k)
  | i == k = s
  | otherwise = Index k
(i -:> s) (Lambda x q t) = Lambda x q (((i + 1) -:> s) ((1-^-1) t))
(i -:> s) (Application r t) = Application ((i -:> s) r) ((i -:> s) t)
(_ -:> _) literal = literal

isValue :: DeBrujinTerm -> Bool
isValue (Application (Lambda _ _ _) _) = False
isValue (Application x y) = (isValue x) && (isValue y)
isValue (If _ _ _) = False
isValue _ = True

typeOf :: Term -> Maybe Type
typeOf term =
  let
    aux :: DeBrujinTerm -> Context -> Maybe Type
    aux (Index k) context = Just (getType k context)
    aux (Lambda x q t) context =
      let
	context' = (x, q) : context
	q' = aux t context'
      in
	case q' of
	  Nothing -> Nothing
	  Just r -> Function q r
    aux (Application t1 t2) context =
      let
	q1' = aux t1 context
	q2' = aux t2 context
      in
	case (q1', q2') of
	  (Nothing, _) -> Nothing
	  (_, Nothing) -> Nothing
	  (Just (Function q11 q12), Just q2) -> if (q11 == q2)
						 then Just q12
						 else Nothing
	  (_, _) -> Nothing
    aux True' _ = Just Boolean
    aux False' _ = Just Boolean
    aux (If t1 t2 t3) context=
      case (aux t1 context) of
	Just q1 -> if (q1 == Boolean)
		   then case (aux t2 context, aux t3 context) of
			  (Just q2, Just q3) -> if (q2 == q3)
						then Just q2
						else Nothing
			  (_, _) -> Nothing
		   else Nothing
	Nothing -> Nothing

    (t, context) = convert term
  in
    aux t context

evaluate :: Term -> (Term, Type)
evaluate term =
  let
    aux :: DeBrujinTerm -> DeBrujinTerm
    aux (Application s t)
      | not (isValue s) = Application (aux s) t
      | not (isValue t) = Application s (aux t)
      | otherwise = case s of
		      Lambda x q e -> ((-1)-^-0) ((0 -:> ((1-^-0) t)) e)
		      _ -> Application s t
    aux (If r s t)
      | predicate == True' = aux s
      | predicate == False' = aux s
      | otherwise = error "Non-boolean in test position!"
      where predicate = aux r
    aux x = x

    (t, context) = convert term
    q' = typeOf term
  in
    case q' of
      Just q -> (q, aux t)
      Nothing -> error "Not a well typed term!"
\end{verbatim}
% Emacs 24.5.1 (Org mode 8.2.10)
\end{document}
